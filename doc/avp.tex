\documentclass{article}
\usepackage{a4}
\usepackage{tabularx}
\newcommand{\degree}{\mbox{${}^{\circ}$}}
\newcommand{\sub}[1]{\kern-.1em \lower.5ex\hbox{\scriptsize #1}}
\newcommand{\super}[1]{\kern-.1em \raisebox{1ex}{\scriptsize #1}}
\newcommand{\AVOL}{\mbox{\AA$^3$}}
\emergencystretch 1in
\let\shortcite\cite
\title{AVP --- Another Void Program}
\author{Andrew C.R. Martin\\
\vspace*{1ex}\\
Institute of Structural and Molecular Biology,\\
Division of Biosciences,\\
University College London,\\
Gower Street,\\
London WC1E 6BT}
\date{14th September, 2004\\Updated 11th December, 2020}

%%%%%%%%%%%%%%%%%%%%%%%%%%%%%%%%%%%%%%%%%%%%%%%%%%%%%%%%%%%%%%%%%%%%%%
\begin{document}
\maketitle

\begin{abstract}
AVP is a new method for the analysis of voids in proteins (defined as
empty cavities not accessible to solvent). AVP combines analysis of
individual discrete voids with analysis of packing quality. While
these are different aspects of the same effect, they have
traditionally been analysed using different approaches.
\end{abstract}


%%%%%%%%%%%%%%%%%%%%%%%%%%%%%%%%%%%%%%%%%%%%%%%%%%%%%%%%%%%%%%%%%%%%%%
\section{Introduction}
Protein cavities are holes in the interior of a protein that are not
accessible to bulk solvent. They may be large enough to accommodate
other atoms or molecules such as water, but may be empty. Empty
cavities are termed `voids'.

Traditionally, a distinction has been made between analysis of
individual voids (normally considered to be those cavities which,
while empty, are large enough to contain a water molecule) and general
packing quality.  In reality, identifying discrete voids and analysing
packing quality are simply extremes of the same phenomenon ---
non-optimal packing of atoms in a protein.

Methods for assessing individual voids include VOIDOO by Kleywegt and
Jones\shortcite{kleywegt:voidoo}, a grid-based program, which maps a
protein onto a 3D grid and then assigns grid points as protein, bulk
solvent, or cavity depending on their location in relation to the
protein structure. In comparison, VOLBL\cite{edelsbrunner:voids} is an
analytical method that utilizes `alpha-shapes' to calculate internal
cavities for both the solvent accessible and molecular surface
models. Both VOIDOO and VOLBL use probes of the same size to detect
the solvent accessible surface of a protein and its internal cavities.
Such methods do not allow discrete voids smaller than the size of a
water molecule to be examined --- if one uses too small a probe, the
cavities will `leak' to the bulk solvent.

In the extreme case of a zero-sized probe, one is making an assessment
of packing quality, but if a single zero-sized probe is used for
assessment of voids and of solvent, all cavities in the protein would
be contiguous with the bulk solvent. Traditionally, therefore,
different methods have been used for packing assessment.  A number of
groups have used Voronoi
polyhedra\cite{lesk:folding,ptitsyn:evolution,gerstein:volume,tsai:packing,richards:packing}.
Other methods are the `occluded surface
algorithm'\cite{pattabiraman:occluded} and the rapid algorithm
QPack\cite{gregoret:qpack}.  Other methods are listed by Fleming and
Richards\shortcite{fleming:packing}.

AVP adapts the grid-based void-finding methods to allow detection of
voids of any size and may therefore also be used to assess packing
quality.  This is achieved by using two probes: one to delimit the
solvent accessible regions, the second to identify voids.  Uniquely,
this method allows for very small or zero sized void probes without
`leakage' of the void into the bulk solvent. We are thus able to
assess both packing quality and individual voids, previously treated
as separate problems, using a single method.

%%%%%%%%%%%%%%%%%%%%%%%%%%%%%%%%%%%%%%%%%%%%%%%%%%%%%%%%%%%%%%%%%%%%%%
\section{Using AVP}

While AVP allows you to use a zero-sized probe for detecting voids, in
practice this is not particularly useful. If you do this to assess the
size of individual voids, then you are most likely to end up with a
single void which `leaks' throughout the protein and is therefore of
similar size to the total void volume. After extensive testing, we
recommend a void probe size of radius 0.5\AA\ which is in general 
too large to pass though gaps between closely packed protein atoms,
but is small enough to detect most important small voids. 

While orientation effects are minimal and measures have been taken in
the program to allow `off-grid' refinements, probe sizes greater than
0.0\AA\ but less than 0.5\AA\ are not recommended since the evaluation
of individual voids is much more sensitive to orientation of the
protein on the grid at these sizes.

Using a zero-sized probe to assess packing is more useful, but, of
course, even optimally packed atoms will show a void volume if this is
done. For example, channels can be seen running down the centre of an
alpha-helix! Once again, extensive testing has shown that a probe
radius of 0.5\AA\ is a sensible choice and has the added benefit of
not having to run the program twice if you are interested in both
individual voids and overall packing.

If one is assessing the effects of mutations on void sizes, then the
fact that a single orientation is used means that relative values can
be compared safely.

Void size refinement must be switched on explicitly with -R and the
void probe size must be specified with -p.


%%%%%%%%%%%%%%%%%%%%%%%%%%%%%%%%%%%%%%%%%%%%%%%%%%%%%%%%%%%%%%%%%%%%%%
\section{Usage}

\noindent {\bfseries Recommended use is:}
\vspace{1em}

\setlength{\fboxrule}{2pt}
\setlength{\fboxsep}{2mm}
\framebox[\linewidth][c]{\tt avp -R -p 0.5 file.pdb file.out}
\vspace{1em}



\begin{verbatim}
avp V1.5 (c) 2001-20, Prof. Andrew C.R. Martin, University of Reading,
                      UCL

Usage: avp [-q] [-g gridspacing] [-p probesize] [-s solventsize]
           [-r] [-e] [-c] [-o[a][s] file] 
           [-f file] [-l] [-n file] [-O(xyz) value] [-w] [-S] 
           [file.pdb [file.out]]
   -q Quiet - do not report progress
   -g Specify the grid spacing (Default: 1.000000)
   -p Specify the probe size (Default: 0.000000)
   -s Specify the solvent size (Default: 1.400000)
   -r Refine void sizes
   -R Redefine void points and refine void sizes
   -e Assign points to voids with edge connections
   -c Assign points to voids with corner connections (implies -e)
   -o Output the void grid points to file. With 'a', also output atom
      grid points; with 's', also output solvent grid points
   -f Output the refined void points to a file. Used with -r
   -l Include protein voxels next to void and solvent in volume 
      refinement
   -n Output atom records for atoms nearest to each void to file.
   -Ox -Oy -Oz Specify an offset for the grid
   -w Print a list of waters that neighbour voids
   -S Reassign surface protein to solvent if can fit a solvent near

avp (Another Void Program) is a program to calculate void volumes in
proteins. It uses a simple grid-based method but separates the probe
size used to define void points as being voids (-p) from the probe
size used to define channels to the surface (-s). Thus one can find
very small voids without these being connected via very small 
diameter passages to the surface.
\end{verbatim}

\begin{description}
\item[-q] The quiet flag simply stops progress information being
  printed. Useful for automated runs.
\item[-g] Specify the grid spacing. Using a finer grid will give more
  accurate evaluations of void sizes, but will dramatically increase
  memory usage. It is rarely necessary to change the
  default. (Default: 1.000000) 
\item[-p] Specify the probe size. This is the most useful parameter
  that you are likely to need to change. The default is only
  recommended for use in packing evaluation; a value of 0.5 is
  recommended for individual void analysis and is also suitable for
  packing evaluation. (Default: 0.000000)
\item[-s] Specify the solvent size. Rarely necessary to change the
  default value which is the size of a water molecule. (Default: 1.400000)
\item[-r] Refine void sizes
  This (or \verb|-R|) is absolutely required. Without it you will only get a
  rough approximation of the void size. Refinement of the void sizes
  splits each voxel into 1000 sub-voxels to fine-tune the void size.
\item[-R] This does a \verb|-r|, but also provides off-grid refinements to
  minimize the effect of grid orientation. This is the recommended default.
\item[-e] Void clustering, by default, only clusters voxels if they
  are face connected. This adds edge-connected voxels. Whether you use
  this depends on your definition of a distinct void!
\item[-c] Assign points to voids with corner connections. As \verb|-e|, but
  also adds corner connected voxels.
\item[-o] Output the void grid points to a PDB file. If you use \verb|-oa|, then
  it will also output grid points assigned as atoms; if you use \verb|-os|,
  then points assigned as solvent will also be output. (You can also
  do -oas). The resulting output file is described in
  Section~\ref{sec:voidsfile}.
\item[-f] Output the refined void points to a file (i.e.\ the 1000 per
  original voxel). Used with \verb|-r| or \verb|-R|. This is really just for
  debugging the void refinement process since there are too many
  points to use effectively.
\item[-l] Include protein voxels next to void and solvent in volume 
  refinement. The voxel refinement normally includes only the voxels
  initially defined as void. Including this option also looks at the
  adjacent protein voxels (which may be part void). This is useful
  with \verb|-r|, but the additional off-grid refinement provided by \verb|-R| means
  it is less useful.
\item[-n] Output atom records for atoms nearest to each void to
  file. This is only partially implemented, but prints the atoms which
  are adjacent to the voids.
\item[-Ox -Oy -Oz] Specify an offset for the grid. This is only really
  used for debugging and analysis of how the program works. It allows
  the grid to be offset on the protein to assess grid orientation
  effects. 
\item[-w] Print a list of waters that neighbour voids. This is used
  only for debugging.
\item[-S] Reassign surface protein to solvent if can fit a solvent
  near. This is the second phase of off-grid refinement. However,
  testing has shown that it does not add much to the lack of
  grid-orientation sensitivity compared with \verb|-R|.
\end{description}
\vspace{1em}

\noindent {\bfseries Recommended use is:}
\vspace{1em}

\setlength{\fboxrule}{2pt}
\setlength{\fboxsep}{2mm}
\framebox[\linewidth][c]{\tt avp -R -p 0.5 file.pdb file.out}
\vspace{1em}


%%%%%%%%%%%%%%%%%%%%%%%%%%%%%%%%%%%%%%%%%%%%%%%%%%%%%%%%%%%%%%%%%%%%%%
\subsection{Void point output file}
\label{sec:voidsfile}

If you use the \verb|-o| option, you will obtain a PDB file containing
the void points. Typically you would merge this with the original PDB
file (\verb|cat original.pdb voids.pdb > both.pdb|) and view the
combined file with something like RasMol or PyMol, rendering the voids
as spacefilled spheres.

The following chain labels are used:
\vspace{1em}

\noindent\begin{tabularx}{\textwidth}{lX}
  X & `Surface voids' --- these are void points next to the water accessible
      surface, but not forming a channel to water. \\
  Y & Fully buried void points. \\
  Z & When using \verb|-oa|, Atom grid points. \\
    & When using \verb|-os|, Solvent grid points. \\
\end{tabularx}


%%%%%%%%%%%%%%%%%%%%%%%%%%%%%%%%%%%%%%%%%%%%%%%%%%%%%%%%%%%%%%%%%%%%%%
\section{Installation}
Unpack the distribution file:

\begin{verbatim}
gzip -d AVP1.5.tar.gz
tar xvf AVP1.5.tar
\end{verbatim}

If you are using GNU tar, you can combine these two steps:

\begin{verbatim}
tar zxvf AVP1.5.tar.gz
\end{verbatim}

This will create a directory, AVP1.5

Now enter the src sub-directory and type make to build the software:

\begin{verbatim}
cd AVP1.5/src
make
\end{verbatim}

With recent versions of the GCC C~compiler, you can also compile to
use multiple cores with OpenMP:

\begin{verbatim}
cd AVP1.5/src
make -f Makefile_gcc.omp
\end{verbatim}


Finally copy the executable to somewhere appropriate:

\begin{verbatim}
cp avp /usr/local/bin
\end{verbatim}

To get help on using AVP, type

\begin{verbatim}
avp -h
\end{verbatim}



%%%%%%%%%%%%%%%%%%%%%%%%%%%%%%%%%%%%%%%%%%%%%%%%%%%%%%%%%%%%%%%%%%%%%%
\section{Algorithm}

The AVP algorithm proceeds as follows:
\begin{description}

\item[Grid construction] A grid is constructed around the protein;
this grid is large enough that at least one plane of water probes can be
placed on all sides. Initially each point on the grid is assigned as
being of type void. By default, the spacing of this grid is 1\AA.

\item[Protein assignment] The grid is searched and any grid point that
is within an atom sphere is changed to type protein. In order to speed
this process, the list of atoms is first sorted along the
$x$-axis. When grid points are checked, a maximum and minimum possible
$x$-coordinate are calculated from the current $x$ grid position plus
or minus the maximum radius of a protein atom (1.9\AA). A binary
search of the sorted atom list is performed to find only those atoms
within a $yz$ slice of the protein neighbouring the required
$x$-coordinate. Simple maximum distance checks are made on the $y$-
and $z$-coordinates before calculation of actual distances.  Also,
while performing this `walk' across the grid, a list of atoms within 2
atom radii of each grid point is associated with that point. This is
used later during void volume refinement (see below).

\item[Solvent assignment] The 6 surfaces of the grid are all assigned
as type water. By moving along the positive and negative $x$-, $y$-
and $z$-axes in turn, each point currently assigned as void is
converted to solvent if a solvent probe can be placed at that point
without clashing with protein and at least one of the 26 neighbouring
points is already solvent (i.e.\ 6 face-connected, 12 edge-connected
and 8 corner-connected neighbours). The same optimization of using an
atom list sorted on $x$-coordinates with a binary search, described
for protein assignment, is used during this stage. As well as setting
each primary grid point to type solvent, any other points within the
solvent radius (optionally multiplied by a `solvent expansion factor')
are also converted to type solvent. This process then iterates until
no new points are converted from void to solvent.  Points within a
solvent radius must be converted to solvent as, if this is not done, a
`shell' of void points is formed all around the protein surface. In
the strategy adopted by Kelywegt and Jones\shortcite{kleywegt:voidoo},
these points were instead assigned as protein as they grew the protein
atoms by the probe radius. An alternative to the iterative strategy
adopted here would be to use a flood fill in 3 dimensions. However,
the standard flood-fill algorithm is recursive and is computationally
impractical for problems of this size.

\item[Void clustering] The preceeding steps have flagged grid points
classified as protein or solvent; remaining points, still assigned as
type void, are now true voids. These are clustered to determine the
distinct void regions in the protein. This is done by walking along
the grid to find void points. Once a void point is found, a standard
three-dimensional flood fill is started to cluster all connected
adjacent points into the same void. The walk across the grid then
continues until another void point is found that is not yet assigned
to a void cluster.

\item[Void volume refinement] At this stage an estimate of the void
volumes may be made by assuming each void point represents a voxel of
volume equal to the grid spacing cubed. At small or zero void probe
sizes this will generally be a large over-estimate although some grid
points may have been assigned as protein whereas the voxel they
represent may be partially void. To improve accuracy, each void voxel
and all neighbouring protein voxels are therefore split into 1000
sub-voxels and the total number of void sub-voxels is then
counted. Only those atoms in the list associated with each original
grid point are examined when making this assignment to speed up the
search. 
\end{description}

The following adaptations are used to minimize the grid orientation
effects, As described above, voxels are initially assigned as protein
by checking whether the centre of the voxel is within the van der
Waals radius of a protein atom.  We then check each voxel, initially
assigned as protein, to see whether it contains any off-centre points
at which a probe of minimum radius 0.1\AA\ could fit. To do this, we
build a list of neighbouring atoms and look at each pair of atoms in
turn. If two atoms are separated by more than the sum of their radii
plus the diameter of the probe, then we walk along the vector between
the two atoms in 0.05\AA\ steps and check whether any point is at
least a probe radius away from all other atoms --- if so, the voxel is
reassigned as void. A voxel can thus be assigned as void even if the
grid spacing assigns it as protein.  In the last step of the procedure
described above, every voxel assigned as void is refined by splitting
it into 1000 sub-voxels each of which is individually assigned as
protein or void.

Secondly we examine all surface protein voxels and, in a similar way,
reassign them as solvent if a solvent probe can be placed at any point
along the vector between the centre of this voxel and the centre of an
adjacent protein voxel. Again, this allows voxels whose centres are
within the protein to be treated as solvent and, after voxel
refinement (which reassigns sub-voxels as solvent or protein) produces
a more accurate evaluation of void volume reducing leakage of voids to
the protein surface.

%%%%%%%%%%%%%%%%%%%%%%%%%%%%%%%%%%%%%%%%%%%%%%%%%%%%%%%%%%%%%%%%%%%%%%
\section{Citing AVP}

If you use AVP in your research, please cite the following paper:
\vspace{1em}

\noindent Cuff, A. L. \& Martin, A. C. R. (2004) Analysis of void
volumes in proteins and application to stability of the p53 tumour
suppressor protein. \emph{J. Mol. Biol.} {\bfseries 344}, 1199--1209.


%%%%%%%%%%%%%%%%%%%%%%%%%%%%%%%%%%%%%%%%%%%%%%%%%%%%%%%%%%%%%%%%%%%%%% 

\bibliographystyle{jmb}
\bibliography{abbrev,lit2}

\end{document}



